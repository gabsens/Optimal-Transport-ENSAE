\documentclass{article}
\usepackage[utf8]{inputenc}
\usepackage[T1]{fontenc}
\usepackage{titlesec}
\usepackage{amsmath}
\usepackage{amsfonts}
\usepackage{amssymb}
\usepackage{amsthm}
\usepackage{enumitem}
\usepackage{bbm}
\usepackage{geometry}
\usepackage{upgreek}
\usepackage{latexsym}
\usepackage{datetime}
%\usepackage{cancel}
\geometry{
 a4paper,
 left=30mm,
 top=20mm,
 }
\setlength{\parindent}{0pt}
\newtheorem{thm}{Theorem}[section]
\newtheorem{corol}{Corollary}[section]
\newtheorem{lemme}{Lemma}[section]
\newtheorem{prop}{Proposition}[section]
\theoremstyle{definition}
\newtheorem{defi}{Definition}[section]
\theoremstyle{remark}
\newtheorem{rem}{Remark}

\title{Notes on Optimal Transport\\ ENSAE\\ 3A}
\author{Gabriel Romon}
\date{Last updated: \today \, at \currenttime}

\DeclareMathOperator{\argmin}{argmin}
\DeclareMathOperator{\argmax}{argmax}
\DeclareMathOperator{\kl}{KL}
\DeclareMathOperator{\data}{data}
\DeclareMathOperator{\id}{Id}


\begin{document}


\maketitle

\section{Introduction}

These are notes I gathered from a seminar session given by Shuangjian Zhang at ENSAE and from lectures given by Marco Cuturi at MLSS 2019 in South Africa. 
\\\\
Likelihood maximization is an instance of generative modeling: given data points $x_1,\ldots,x_N\in \mathbb R^d$ and a family of densities $(f_\theta)_{\theta \in \Theta}$, we look for the $f_\theta$ that matches the most the empirical data distribution defined as $\nu_{\data}:=\frac 1N \sum_{i=1}^N \delta_{x_i}$. This is done by maximizing the log-likelihood $\frac 1N \sum_{i=1}^N \log f_\theta(x_i)$. This quantity exists only if $f_\theta(x_i)>0$ for all $i$, which forces the densities to have the whole of $\mathbb R^d$ as support. Likelihood maximization can be given a geometric interpretation: if one overlooks that $\nu_{\data}$ and $f_\theta$ are not absolutely continuous with respect to the same measure (the former is discrete), minimizing the Kullback-Leibler divergence between the two writes as $\begin{aligned}[t]\argmin_{\theta\in \Theta} \kl(\nu_{\data}||f_\theta) &= \argmin_{\theta\in \Theta} E_{X\sim \nu_{\data}}[\log(f_{\data}(X))-\log(f_\theta(X))]\\
&= \argmin_{\theta\in \Theta} -E_{X\sim \nu_{\data}}\log(f_\theta(X))\\
&= \argmax_{\theta\in \Theta} \frac 1N \sum_{i=1}^N \log f_\theta(x_i)
\end{aligned}$
\\\\\\
A weakness of likelihood maximization is its poor scalability to high-dimensional settings, which are commonplace. For instance, a $100\times 100$ image with 3 color channels lives in $\mathbb R^{30.000}$. Instead of working directly in the data space $\mathbb R^d$, we may rather consider a latent space $(\mathbb R^p, \mathcal B(\mathbb R^p), \mu)$ with $p\ll d$, and measurable functions $g_\theta:\mathbb R^p\to \mathbb R^d$ (e.g deconvolution networks). We define the pushforward measure $g_{\theta \sharp}\mu$ by $\forall B\in \mathcal B(\mathbb R^d), g_{\theta \sharp}\mu(B) := \mu(g_{\theta} \in B)$ and we look for $\theta$ such that $g_{\theta \sharp}\mu$ matches $\nu_{\data}$. This requires setting a metric on the space of probability measures, and fortunately, many exist: Hellinger, Kantorovitch, MMD, Wasserstein... Some of these metrics arise from the theory of optimal transport.

\section{Optimal transport}

\subsection{Monge problem and its Kantorovitch relaxation}

Let $\mu$ and $\nu$ be probability measures on $(\mathbb R^d, \mathcal B(\mathbb R^d))$ and $c:\mathbb R^d\times \mathbb R^d\to \mathbb R_+$ a cost function. Monge's optimal transportation problem is the following:
\begin{equation}\tag{MP}
\inf_{\substack{T \text{ measurable}\\ T_\sharp \mu = \nu}} \int c(x,T(x)) d\mu(x) 
\end{equation}

In words, $T$ is a transportation mapping that assigns to each $x$ exactly one location $T(x)$, implying that the mass at $x$ cannot be split to several locations. $T_\sharp \mu = \nu$ is the transportation constraint, and the objective $\int c(x,T(x)) d\mu(x) $ measures the total transportation cost.
\\ \\
The Kantorovitch relaxation allows for mass splitting by considering couplings. A probability measure $P$ on $\mathbb R^d \times \mathbb R^d$ is said to be a coupling of $(\mu,\nu)$ if for any $A,B \in \mathcal B(\mathbb R^d), P(A\times \mathbb R^d) = \mu(A)$ and $P(\mathbb R^d\times B) = \nu(B)$. Let $\Pi(\mu, \nu)$ denote the set of such couplings. $\Pi(\mu, \nu)$ is clearly non-empty since it contains the product measure $\mu \otimes \nu$.\\
Kantorovitch's optimal transportation problem is the following:
\begin{equation}\tag{KP-primal}
\inf_{P\in \Pi(\mu, \nu)} \int\int c(x,y) dP(x,y)
\end{equation}

It is important to note that (KP-primal) provides a lower bound for (MP). Indeed, if $T$ verifies $T_\sharp \mu = \nu$ and we let $P=(\id,T)_{\sharp}\mu$, it is easy to check that $P$ is a coupling. A theorem of integration with respect to pushforward measures yields $\int \int c(x,y) dP(x,y) = \int c\circ (\id,T)(x) d\mu(x) = \int c(x,T(x)) d\mu(x)$, hence the claim. %add reference to later connection ?
\\ 
\\
The dual of (KP-primal) is defined as follows:
\begin{equation}\tag{KP-dual}
\sup_{\substack{\varphi \in L_1(\mu)\\ \psi \in L_1(\nu)\\ \forall (x,y), \;\varphi(x)+\psi(y)\leq c(x,y)}} \int \varphi d\mu + \int \psi d\nu
\end{equation}

\begin{thm}
If $c$ is lower semi-continuous, then (KP-primal) and (KP-dual) share the same optimal value. Moreover, the infimum in (KP-primal) is attained.
\end{thm}

\begin{proof}
We give a partial proof of the first part of the theorem. Let $\varphi\oplus \psi:(x,y) \mapsto \varphi(x)+\psi(x)$ and $$\iota_\Pi(P) := \sup_{\substack{\varphi \in L_1(\mu)\\ \psi \in L_1(\nu)}} \int \varphi d\mu + \int \psi d\nu - \int \varphi\oplus \psi dP$$
Let $P\in \Pi(\mu, \nu)$. If we let $\pi_1$ and $\pi_2$ denote the projections respectively on the first and last $d$ coordinates, then $\int \varphi\oplus \psi dP = \int \varphi \circ \pi_1 dP + \int \psi \circ \pi_2 dP$. For $\varphi = 1_A$, $$\int \varphi \circ \pi_1 dP = \int 1_{\pi_1^{-1}(A)} dP = P(\pi_1^{-1}(A)) = P(A\times \mathbb R^d) = \mu(A)$$
Therefore $\int \varphi d\mu + \int \psi d\nu - \int \varphi\oplus \psi dP = 0$ when $\varphi$ and $\psi$ are indicators, and a standard limit argument shows that it remains $0$ for arbitrary integrable $\varphi$ and $\psi$. Thus $\iota_\Pi(P)=0$ when $P$ is a coupling.\\
If $P\notin \Pi(\mu, \nu)$, WLOG there exists some $A\in \mathcal B(\mathbb R^d)$ such that $P(A\times \mathbb R^d)\neq \mu(A)$. If $P(A\times \mathbb R^d)< \mu(A)$, consider $\varphi = \lambda 1_A$ with $\lambda >0$ and $\psi = 0$. Then $$\int \varphi d\mu + \int \psi d\nu - \int \varphi\oplus \psi dP = \lambda (\mu(A) - P(A\times \mathbb R^d)) \xrightarrow[\lambda \to \infty]{} \infty$$ Hence $\iota_\Pi(P)=\infty$. The other case is similar.\\
\\
If we let $\mathcal M^+$ denote the set of positive measures on $\mathbb R^d\times \mathbb R^d$, then (KP-primal) turns into 
$$\begin{aligned}
	\inf_{P\in \mathcal M^+} \int\int c\; dP +  \iota_\Pi(P) 
	&= \inf_{P\in \mathcal M^+} \sup_{\substack{\varphi \in L_1(\mu)\\ \psi \in L_1(\nu)}}  \int c\; dP + \int \varphi d\mu + \int \psi d\nu - \int \varphi\oplus \psi dP \\
	&= \inf_{P\in \mathcal M^+} \sup_{\substack{\varphi \in L_1(\mu)\\ \psi \in L_1(\nu)}}  \int c - \varphi\oplus \psi\; dP + \int \varphi d\mu + \int \psi d\nu \\
	&= \sup_{\substack{\varphi \in L_1(\mu)\\ \psi \in L_1(\nu)}} \inf_{P\in \mathcal M^+}   \left[\int c - \varphi\oplus \psi\; dP\right] + \int \varphi d\mu + \int \psi d\nu
\end{aligned}$$
Switching the $\inf$ and the $\sup$ is the technical hurdle of the proof. In \cite{villani2003topics} Villani provides a rigorous justification that is quite involved. \\
Next, $\inf_{P\in \mathcal M^+}   \int c - \varphi\oplus \psi\; dP$ can be rewritten more simply. If $c - \varphi\oplus \psi \geq 0$, $0$ is a lower bound for the integral, and it is attained for $P=0$. Hence $c - \varphi\oplus \psi \geq 0 \implies \inf_{P\in \mathcal M^+}   \int c - \varphi\oplus \psi\; dP = 0$. Otherwise, if there exists $(x_0,y_0)$ such that $c(x_0,y_0) - \varphi\oplus \psi(x_0,y_0) <0$, consider $P=\lambda \delta_{(x_0,y_0)}$ and let $\lambda \to \infty$ to get $\inf_{P\in \mathcal M^+}   \int c - \varphi\oplus \psi\; dP = -\infty$.\\
Thus (KP-primal) can be written as $$\sup_{\substack{\varphi \in L_1(\mu)\\ \psi \in L_1(\nu)\\ \varphi \oplus \psi \leq c}} \int \varphi d\mu + \int \psi d\nu$$
which is exactly (KP-dual).
\end{proof}

\subsection{$c$-transforms}

The constraint in (KP-dual) rewrites as $\forall x,y \in \mathbb R^d, \psi(y)\leq c(x,y) - \varphi(x)$ or equivalently \newline $\forall y,\; \psi(y) \leq \inf_x [c(x,y) - \varphi(x)]$. This motivates the definition of the $c$-transform of $\varphi$ as $$\varphi^c(y):= \inf_x  c(x,y) - \varphi(x)$$ and the the $c$-transform of $\psi$ as $$\psi^c(x):= \inf_y  c(x,y) - \psi(y)$$ However, $\varphi^c$ and $\psi^c$ may not be integrable without additional hypotheses on $c$, so we assume as in Exercise 2.36 of \cite{villani2003topics} that there exist $c_X\in L_1(\mu)$ and $c_Y\in L_1(\nu)$ such that $\forall x,y \in \mathbb R^d, c(x,y)\leq c_X(x) + c_Y(y)$. With this assumption it is easily checked that $\varphi^c$ is $\nu$-integrable and the pair $(\varphi, \varphi^c)$ increases the objective function. The same argument applies to $(\varphi^{cc}, \varphi^c)$, so (KP-dual) rewrites as 
\begin{equation}\tag{KP-dual 2}
\sup_{\varphi \in L_1(\mu)} \int \varphi d\mu + \int \varphi^c d\nu
\end{equation}
and 
\begin{equation}\tag{KP-dual 3}
\sup_{\varphi \in L_1(\mu)} \int \varphi^{cc} d\mu + \int \varphi^c d\nu
\end{equation}
\\ \\
It is important to note that $\varphi^{ccc}=\varphi^c$. 
\begin{proof}
Note that $\varphi\leq \varphi^{cc}$. Indeed for $x,y\in \mathbb R^d$, $$\begin{aligned}
\varphi^c(y) &\leq c(x,y) - \varphi(x)\\
\implies \varphi(x) &\leq c(x,y) - \varphi^c(y) \quad \forall y\\
\implies \varphi(x) &\leq \inf _y c(x,y) - \varphi^c(y)\\
\implies \varphi(x) &\leq \varphi^{cc}(x)
\end{aligned}$$
Next, for $x,y\in \mathbb R^d$,
$$\begin{aligned}
\varphi^{ccc}(y) &\leq c(x,y) - \varphi^{cc}(x)\\
\implies \varphi^{ccc}(y) &\leq c(x,y) - \varphi(x) \quad \forall x\\
\implies \varphi^{ccc}(y) &\leq \inf_x c(x,y) - \varphi(x)\\
\implies \varphi^{ccc}(y) &\leq \varphi^c(y)
\end{aligned}$$
and 
$$\begin{aligned}
\varphi^{cc}(x) &\leq c(x,y) - \varphi^{c}(y)\\
\implies \varphi^{c}(y) &\leq c(x,y) - \varphi^{cc}(x) \quad \forall x\\
\implies \varphi^{c}(y) &\leq \inf_x c(x,y) - \varphi^{cc}(x)\\
\implies \varphi^{c}(y) &\leq \varphi^{ccc}(y)
\end{aligned}$$
\end{proof}

Besides, $\varphi$ is said to be $c$-concave if $\varphi^{cc}=\varphi$. In \cite{villani2003topics}, Villani states that (KP-dual) is solved by a pair $(\varphi, \varphi^c)$ where $\varphi$ is $c$-concave. As a result, (KP-dual) may be also rewritten as 
\begin{equation}\tag{KP-dual 2}
\sup_{\substack{\varphi \in L_1(\mu)\\ \varphi\; c-\text{concave}}} \int \varphi d\mu + \int \varphi^c d\nu
\end{equation}

\subsection{Wasserstein distances}

\newpage
\bibliographystyle{unsrt}
\bibliography{bibli}

\end{document}